%Vorlage fuer Thesen an der FFHS
\documentclass{../template/ffhsthesis}

\usepackage[utf8]{inputenc}

% addional configuration from sam
\usepackage{import}
\usepackage{bibgerm}
\usepackage{hyperref}
\usepackage{cite}





\begin{document}
% addional configuration from sam
\shorthandoff{"}

\dokumentTyp{Disposition für die Bachelor-Thesis}
\studiengang{INF}
\title{Energie in der Informatik}
\subtitle{ durch bessere Software sparen ohne Verzicht} % optional
\titelbild[height=4.55cm,width=15cm]{1-iPhone-5-Wallpaper-Panorama-Earth-Sun-Space.jpg}  % optional
\author{Samuel Riolo}
% \date{}
\wohnort{Kerzers}
%\referent{Name des Referenten\\ Titel\\Unterrichtetes Fach}
\referentin{Jürg Hofer\\ Titel\\Unterrichtetes Fach}
\eingereichtBei{Prof.\ Dr.\ Martin Sutter\\Departement Informatik\\Departementsleiter} 


\maketitle



\tableofcontents


\begin{abkuerzungen}[MUSTER] % Das Muster dient zur Bestimmung der Einrueckungstiefe
\item[DInf] Departement Informatik
\item[FFHS] Fernfachhochschule Schweiz
\end{abkuerzungen}


\startThesis % Befehl muss vor dem ersten chapter stehen (Seitennummerierung!)


% addional configuration from sam
\addtolength{\parskip}{\baselineskip}
\parindent 0pt 



% content
\subimport*{sections/}{einleitung}
\subimport*{sections/}{wissenschaftlicherbezug}
\subimport*{sections/}{fragestellung}
\subimport*{sections/}{vorgehen}
\subimport*{sections/}{resultate}




%\begin{thebibliography}{1}
%\end{thebibliography}
\bibliography{thsis_bibliography}
\bibliographystyle{gerplain}


% oder besser




\end{document}


