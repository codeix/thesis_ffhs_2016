%Vorlage fuer Thesen an der FFHS
\documentclass{../template/ffhsthesis}

\usepackage[utf8]{inputenc}

% addional configuration from sam
\usepackage{bibgerm}
\usepackage{hyperref}
\usepackage{cite}
\usepackage{etoolbox}
\makeatletter
\patchcmd{\chapter}{\if@openright\cleardoublepage\else\clearpage\fi}{}{}{}
\makeatother




\begin{document}
% addional configuration from sam
\shorthandoff{"}

\dokumentTyp{Disposition für die Bachelor-Thesis}
\studiengang{INF}
\title{Energie in der Informatik}
\subtitle{ durch bessere Software sparen ohne Verzicht} % optional
\titelbild[height=4.55cm,width=15cm]{Italy_Alps_and_Mediterranean.jpg}  % optional
\author{Samuel Riolo}
% \date{}
\wohnort{Kerzers}
%\referent{Name des Referenten\\ Titel\\Unterrichtetes Fach}
\referent{Jürg Hofer\\ Titel\\Unterrichtetes Fach}
\eingereichtBei{Prof.\ Dr.\ Martin Sutter\\Departement Informatik\\Departementsleiter} 


\maketitle



\tableofcontents


\begin{abkuerzungen}[MUSTER] % Das Muster dient zur Bestimmung der Einrueckungstiefe
%\item[DInf] Departement Informatik
%\item[FFHS] Fernfachhochschule Schweiz
\item[SoC] System on Chip
\item[RISC] Reduced Instruction Set Computer
\item[CISC] Custom Instruction Set Computer
\item[ARM] Advanced RISC Machine
\item[EpI] Energy per Instruction

\end{abkuerzungen}


\startThesis % Befehl muss vor dem ersten chapter stehen (Seitennummerierung!)


% addional configuration from sam
\addtolength{\parskip}{\baselineskip}
\parindent 0pt 



% content
\chapter{Einleitung}


In der heutigen Zeit ist die nachhaltige Produktion von Energie und ein schonender Umgang mit dieser Ressource ein wichtiges Anliegen der Öffentlichkeit. Der Bundesrat hat einen Teil der Ziele seiner Energiepolitik in dem Partnerschaftsprogramm EnergieSchweiz festgelegt. Aus Sicht der Informatikbranche ist es interessant zu evaluieren, welchen Anteil sie zur Energieeinsparung beitragen kann. Aus der heutigen Gesellschaft sind Computer nicht mehr wegzudenken. Sie sind sowohl im privaten wie im geschäftlichen Bereich weit verbreitet und erleben nach wie vor einen starken Zuwachs. Daher ergibt auch die kleinste Einsparung pro Computer in der Summe ein immenses Energiesparpotenzial und einen beachtlichen quantitativen Anteil am gesamthaften Sparpotenzial. 
\par
In der IT-Branche existieren bereits heute zahlreiche Möglichkeiten, mit welchen ein nachhaltiger und schonender Umgang mit elektrischer Energie erreicht werden kann. Aus allen möglichen Massnahmen zur Energieeinsparung in der Computerwelt wurde hier jene ausgewählt und erforscht, die auf bereits bestehender Hardware umgesetzt werden kann und somit keinen Ersatz dieser erfordert. Die Basis aller Computertätigkeiten ist die Abarbeitung von unendlich vielen Befehlssätzen durch den Prozessor. Selbst Programme, die interpretiert werden oder in einer höheren Programmiersprachen geschrieben wurden, werden schliesslich als eine Reihe von Befehlssätzen auf dem Prozessor ausgeführt. Aus diesem Grundwissen wird folgende Forschungsfrage formuliert: Ist es möglich, durch effizientere Befehlssätze Energie zu sparen?
\par
Um diese Frage zu beantworten, wurde im Rahmen dieser Engineeringarbeit eine Messmethodik zur Analyse des Energiebedarfs eines Prozessors pro Befehlssatz entwickelt. Sie ermöglicht es, den Energiebedarf einzelner, beliebiger Befehlssätze effizient zu messen. So können Daten erstellt, die Informationen über den Energieverbrauch einzelner Befehlssätze liefern. Die Messungen erfolgen dabei auf unterschiedlichen CPU-Architekturen, damit abweichende Messresultate beobachtet werden können.
\par
Im Endeffekt ist das Hauptziel der vorgenommenen Messungen die Evaluation des Energiesparpotentials. Mit dieser Arbeit soll aufgezeigt werden, ob im Softwarebereich Energie gespart werden kann. Gleichzeitig sollen die Messmethodik und die daraus gewonnenen Erkenntnisse als Grundlage für weitergehende Forschungen dienen. So könnten diese beispielsweise zur Optimierung von Smartphones im Sinne einer längeren Akkulaufzeit eingesetzt werden.
\par
Um die technische Aspekte für den Leser nachvollziehbar zu gestalten, wurde mit viele Grafiken gearbeitet. Alle Abbildungen in dieser Arbeit wurden vom Autor selbst erstellt \footnote{Ausgenommen die Produktbilder in den Abbildungen \ref{fig:Intel Galileo Gen2} und \ref{fig:Raspberry Pi 1 A}.}. 












\begin{comment}

In der heutigen Zeit ist die nachhaltige Produktion von Energie und ein schonender Umgang mit dieser Ressource ein wichtiges Anliegen der Öffentlichkeit. Speziell die IT-Branche ist von einem grossen Wachstum geprägt. Immer mehr Tätigkeiten werden automatisiert und bestehende Systeme ausgebaut. Dies verursacht einen immer grösseren Stromverbrauch im IT-Bereich. Ein Ziel der Energiepolitik 2050 ist die Senkung des immer weiter steigenden Energieverbrauchs.
\par
Um in der IT-Branche Energie zu sparen, können verschiedene Ansätze gewählt werden. Beispielsweise besteht die Möglichkeit bei Inaktivität automatisch in den Ruhemodus zu wechseln oder die Prozessoren durch sogenannte Sparmodi, insbesondere dynamische Taktfrequenzen oder das kurzzeitige Ein- und Ausschalten der Rechnereinheit, zu optimieren. Diese Sparansätze zielen jeweils nur auf den Hard- oder Softwarebereich ab. Sie berücksichtigen jedoch nicht die besondere Funktionsweise eines jeden Computers, wo die Software die Hardware, also ein Programm einen Prozessor, ansteuert. Anders im Rahmen dieser Arbeit, in welcher das Sparpotenzial bereichsübergreifend erforscht wird.



\section{Übersicht}

Die Energiepolitik ist auf der ganzen Welt ein grosses Thema und wird heftig diskutiert. Immer mehr Leute
sind der Überzeugung, dass eine Energiewende unumgänglich ist. Dabei wird die Energiewende von politischen, aber auch
von wirtschaftlichen Interessen beeinflusst. Die Energiewende beinhaltet aber nicht nur
das Produzieren von erneuerbaren Energien, sondern viele unterschiedliche Teilaspekte. Für die Energiewende
müssen Energieressourcen gespeichert werden, damit sie bei Bedarf genutzt werden können. Wird Strom über
Solaaranlagen in den Haushalten produziert, so muss die Energie ins Netz zurückfliessen. Dafür müssen
Hochspannungsleitungen so verändert werden, dass sie Strom in beide Richtungen fliessen lassen können.
Die heute verwendeten Energienanlagen, wie Windmühlen oder Solaaranlagen, produzieren in Europa so starke
Schwankungen, dass die Gefahr eines riesigen Blackouts besteht.
\par
Viele Probleme müssen für die Energiewende gelöst werden. Eines der wichtigsten Kriterien der Energiewende ist aber,
dass Energie gespart wird. Diese Arbeit wird sich mit einem Teilgebiet der Energiewende befassen.
Es soll aufgezeigt werden, wo in der IT-Branche, speziell im Client- und Serverbereich, Potenzial besteht,
Strom zu sparen. 

% todo Überarbeiten, ausführlicher
Auch gerade deswegen, weil die IT-Branche von einem enormen Wachstum geprägt ist. Wirtschaftliche Aspekte sollen
berücksichtigt werden, indem der Erfolg des Stromsparens ohne Verzicht auf die Qualität der IT-Infrastruktur erzielt
werden kann. 
\par
% todo Zusammenfassung der Arbeit, nicht Beschreibung, was ich machen werde
Ein Computer ist ein Gerät, dessen Besonderheit darin besteht, dass es durch logische und arithmetische Befehle programmiert
werden kann. In andere Worte gefasst, heisst das, dass die Software die Hardware ansteuert und die Hardware Befehl um Befehl ausführt.
Genau dieser Aspekt soll in der Arbeit abgehandelt werden. Es soll eine Basis gebildet werden, die aufzeigt, wie man durch bessere Software Strom
sparen kann. Die Arbeit wird sich darauf konzentrieren, den Energiebedarf pro Befehlssatz eines Prozessors zu analysieren.
Es soll eine Messmethoden erforscht und entwickelt werden, um den Strombedarf eines CPU zu untersuchen. Die Untersuchung soll
auf unterschiedlichen SoC (System on Chip) Architekturen erfolgen, damit Vergleiche erstellt werden können.
\par
Ziel der Arbeit ist es, Daten zu erhalten, die Informationen über den Energieverbrauch einzelner Befehlssätze liefern. Diese Daten
könnte man zum Beispiel auf Workstations oder Server hochrechnen. Für CPU-Befehle, die sich eher verschwenderisch auf den
Strombedarf auswirken, könnte man Alternativen finden.

\end{comment}


% Denn nur so kann eine Aussage gemacht werden, ob ein neues System auch zielführend ist. Erst im zweiten Schritt können dann
% Ansätze evaluiert werden, die zum Ziel der Reduktion des Energieverbrauchs dienen können. 
% Dabei sollen unterschiedliche Schichten der Software analysiert werden. Es sollen Antworten auf die folgenden Fragen gefunden werden:
% Wie kann das Betriebsystem Strom sparen; ist es möglich durch bessere Treiber die Hardware auf Standby zu setzen, wenn sie nicht gebraucht wird
% oder kann bereits beim Kompilieren effizenter gearbeitet werden. Am Schluss soll geprüft werden,
% ob die Ansätze in der Realität anwendbar und wirtschaflich interessant sind. Es sollen konkrete Beispiele enstehen, wo und wie ein
% IT-Betrieb Stromersparnisse erzielen kann.



\chapter{Fragestellung}

\chapter{Vorgehen}

Der Hauptteil dieser Arbeit behandelt die Vorbereitung für die einzelnen Messungen. Um möglichst viele Störfaktoren
von der Messung auszuschliessen, wird auf ein möglichst einfaches System gesetzt. Deshalb kommt ein SoC
(System on Chip) zur Anwendung. Zusätzlich sollen auch die Unregelmässigkeiten des Betriebssystems ausgeschaltet
werden. Aus diesem Grund soll, soweit möglich, das OS ausgelassen werden und direkt auf Bare Metal gearbeitet werden.
Somit lassen sich die Anweisung ganz genau kontrollieren. Für die einzelne Messung wird ein kleines Programm geschrieben, das den Prozessor einen bestimmten Befehl mehrmals ausführen lässt. Während der Ausführung wird die Leistung gemessen und ein Durchschnittswert berechnet.
Der SoC ist auf einem Single Board aufgelötet, wobei
die Messung bei dessen Eingangsleistung erfolgt. Für das Resultat der EPI muss die Grundleistung von der Eingangsleistung abgezogen
werden.
\par
Für das Projekt werden zwei Typen von Prozessoren verwendet. Es werden sowohl Messungen mit einem RISC-Prozessor
als auch mit einem CISC-Prozessor durchgeführt. Wo immer möglich, werden die Architekturen der beiden Prozessoren miteinander verglichen
und die sich daraus ergebenden Schlüsse gezogen.
\par
Bei den durchzuführenden Messungen auf dem jeweiligen CPU werden nicht nur die unterschiedlichen Operationen untersucht, sondern ebenso 
Experimente erarbeitet. Es soll zum Beispiel eine Aussage darüber getroffen werden, ob eine Addieroperation
mit grossen Zahlen mehr Energie benötigt, als dieselbe Operation mit kleinen Zahlen.



\chapter{Wissenschaftlicher Bezug}

\section{Energiepolitik 2050}
Beginnend mit einem Überblick über die Energiepolitik 2050, werde ich den Bezug zur Entwicklung in der
Informatik schaffen. Dazu werde ich mich auf die Publizierung
"Energiezukunft Schweiz"\cite{eth_energiezukunft_schweiz} von der ETH sowie auf die\cite{swico_datenblatt}
der Swico stützen.

\section{Geschichte der CPUs}
Im Anschluss möchte ich in meiner Arbeit auch die Geschichte der Prozessoren präsentieren. Hier wird der Bezug
zwischen RISC- und CISC-Prozessoren hergestellt. Hinsichtlich des Verbrauchs spielt die Unterscheidung zwischen den verschiedenen 
Prozessorarchitekturen eine grosse Rolle\cite{stanford_risc_cisc}. In diesem Teil sollen auch beide Architekturen erklärt werden.


\section{Messung}
Für das Experiment müssen Methoden verwendet werden, die den Energiebedarf eines CPUs pro Befehlssatz messen
\cite{measuring_power_temperature, analysis_circuits, intel_epi}. Für eine optimale Messung müssen die bestehenden Methoden
verfeinert oder eigene entwickelt werden.
\par
Für die Messungen werden spezielle Programme geschrieben. Diese führen einen bestimmten Befehlssatz während dem Messen mehrmals aus.
Die auszuführenden Befehlssätze für das Experiment werden anhand des Datasheets des Herstellers zusammengestellt.
Voraussichtlich wird das Experiment auf einem ARM ARM1176JZF-S\cite{arm_datasheet} und einem Intel® Quark™ SoC
X1000 \cite{intel_datasheet} Prozessor ausgeführt.



\chapter{Resultate}

Die Thesis soll eine Messmethode liefern, die Beschreibt wie man der EPI eines CPU Messen kann. Dabei soll auch versucht werden
unterschiedliche CPU-Architekturen zu berücksichtigen. Die Messmethode soll sehr genau beschrieben werden, damit man sie
nachstellen kann und für eigene Zwecke weiter verwenden kann.
\par
Aus der Messmethode sollen in dieser Arbeit interessante Experimente entstehen. Die Daten aus dem Experiment werden ausgewertet
und in einem Fazit dargestellt.
\par
Die Arbeit soll als Grundlage dienen, weiter in diesem Gebiet zu forschen und besser verstehen welche CPU-Anweisungen energieeffizienter
sind. Aus dieser Grundlage soll sich auch die Massnahmen ableiten lassen wie auf der Ebene der Hardware Energie sparen lässt. Es soll
aus diesem Wissen neue Kreative Ideen entstehen, zum Beispiel wie ein Green-Kompiler gebaut werden kann oder ein Energy-Profiler
der bereits bei schreiben der Software markiert welche Code-Zeilen ungeeignet sind.


%Das Resultat meiner Arbeit soll sich nicht nur auf einer theoretische Basis bewegen,
%sonder klar aufzeigen wo Handlungsbedarf besteht. Aus der Arbeit soll sich klare
%Handlungsempfehlungen entnehmen. Die Resultate sollen sich realistisch auf ein IT-Betrieb
%umsetzen lassen. Dabei soll ein IT-Betrieb Energie sparen können, ohne Verzicht auf die Qualität.






%\begin{thebibliography}{1}
%\end{thebibliography}
\bibliography{thesis_bibliography}
\bibliographystyle{gerplain}

% oder besser




\end{document}


