\chapter{Resultate}

Die Thesis soll eine Messmethode liefern, die Beschreibt wie man der EPI eines CPU Messen kann. Dabei soll auch versucht werden
unterschiedliche CPU-Architekturen zu berücksichtigen. Die Messmethode soll sehr genau beschrieben werden, damit man sie
nachstellen kann und für eigene Zwecke weiter verwenden kann.
\par
Aus der Messmethode sollen in dieser Arbeit interessante Experimente entstehen. Die Daten aus dem Experiment werden ausgewertet
und in einem Fazit dargestellt.
\par
Die Arbeit soll als Grundlage dienen, weiter in diesem Gebiet zu forschen und besser verstehen welche CPU-Anweisungen energieeffizienter
sind. Aus dieser Grundlage soll sich auch die Massnahmen ableiten lassen wie auf der Ebene der Hardware Energie sparen lässt. Es soll
aus diesem Wissen neue Kreative Ideen entstehen, zum Beispiel wie ein Green-Kompiler gebaut werden kann oder ein Energy-Profiler
der bereits bei schreiben der Software markiert welche Code-Zeilen ungeeignet sind.


%Das Resultat meiner Arbeit soll sich nicht nur auf einer theoretische Basis bewegen,
%sonder klar aufzeigen wo Handlungsbedarf besteht. Aus der Arbeit soll sich klare
%Handlungsempfehlungen entnehmen. Die Resultate sollen sich realistisch auf ein IT-Betrieb
%umsetzen lassen. Dabei soll ein IT-Betrieb Energie sparen können, ohne Verzicht auf die Qualität.

