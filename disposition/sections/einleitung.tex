\chapter{Einleitung}

Mein Name ist Samuel Riolo geboren am 24.12.1985 und besuche seit 2013 die FFHS. Dieses Jahr werde ich mein Bachleor in Inforamtik
abschliessen. Als Quereinsteiger bin ich Beruflich in die Softwareenticklung eingestiegen. Mir bereitet das
programmieren viel Spass und so wurde die Informatik eine grosse Leidenschaft von mir. Ich möchte mein Referent Jürg
Hofer und mein CO-Referent Walter Brigger danken, dass sie meine Arbeit begleiten.
\par
Die Energiepolitik ist auf der ganze Welt ein grosses Thema und wird heftig diskutiert. Immer mehr Leuten
sind der Überzeugung, dass eine Energiewende unumgäglich ist. Dabei wird die Energiewende von politischen aber auch
aus Wirtschaftlichen interessen, in allen Richtungen bewegt. Die Energiewende beinhaltet aber nicht nur
das Produzieren von erneuerbaren Energien sondern viele unterschiedliche Teilaspekte. Für die Energiewende
müssen Energiequellen gespeichert werden, damit sie bei bedarf genützt werden können. Wird Strom über
Solaaranlagen der Haushalte produziert so muss die Energie ins Netz zurück fliessen. Dafür müssen
Hochspannungsleitungen so genändert werden, dass sie Strom in beide richtungen fliessen lassen  können.
Die heute angewendeten erneuerbare Energien, wie Windmühlen oder Solaaranlagen, produzieren in Europa so starke
Schwankungen, dass die Gefahr besteht von einem riesiegen Blackout.
\par
Viele Probleme müssen für die Energiewende gelöst werden, einer der wichtigen Kriterien der Energiewende ist aber
das Energie gespart wird. Diese Arbeit wird sich aus einem kleinem Stück aus einem riesiegen Gebiet, der Energiewende,
befassen. Es soll aufgezeigt werden wo in der IT-Branche, spezielle im Client und Server bereich, potenzial besteht
Strom zu sparen. Auch gerade deswegen weil die IT eine enorm Wachsende Branche ist. Wirtschaftliche Aspekte sollen
berücksichtigt werden in dem der Erfolg des Stromsparen ohne Verzicht auf die Qualität der IT-Infrastruktur erziehlt
werden kann. 
\par
Die Bezeichnung eines Computer, ist ein Gerät, dass durch Logische und Arithmetische Befehle programmiert werden kann. In
anderen Worten gefasst, heisst das, dass die Software die Hardware ansteuert und die Hardware Befehl um Befehl aufführt.
Um genau diesem Punkt soll sich die Arbeit befassen. Es soll gezeigt werden wie man durch bessere Software Strom
sparen kann. Im ersten Schritt müssen Messmethoden erforscht und dargestellt werden. Denn nur so kann eine Aussage
gemacht werden, ob ein neues System auch Zielführend ist. Erst im zweiten Schritt können dann
Ansätze evaluiert werden, die zum Ziel der reduktion des Energie Verbrauch dienen können. Erforscht soll dabei
unterschiedliche Schichten der Software. Es sollen Antworten auf Fragen enstehen wie, kann das Betriebsystem
Stromsparen, ist es möglich durch bessere Driver die Hardware auf Standby zu setzten wenn sie nicht gebraucht wird oder
kann bereits beim compilieren effizenter gearbeit werden. Am Schluss soll geprüft werden
ob die Ansätze in der Realität anwenbar und wirtschaflich interessant sind. Es sollen konkrete Beispiele enstehen wo ein
IT-Betrieb Strom Ersparnisse erziehlen kann.

