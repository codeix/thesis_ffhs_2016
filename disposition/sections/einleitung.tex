\chapter{Einleitung}

Die Energiepolitik ist auf der ganzen Welt ein grosses Thema und wird heftig diskutiert. Immer mehr Leute
sind der Überzeugung, dass eine Energiewende unumgäglich ist. Dabei wird die Energiewende von politischen, aber auch
aus wirtschaftlichen Interessen beeinflusst. Die Energiewende beinhaltet aber nicht nur
das Produzieren von erneuerbaren Energien, sondern viele unterschiedliche Teilaspekte. Für die Energiewende
müssen Energieressourcen gespeichert werden, damit sie bei Bedarf genutzt werden können. Wird Strom über
Solaaranlagen der Haushalte produziert, so muss die Energie ins Netz zurückfliessen. Dafür müssen
Hochspannungsleitungen so geändert werden, dass sie Strom in beide Richtungen fliessen lassen können.
Die heute verwendeten Energienanlagen, wie Windmühlen oder Solaaranlagen, produzieren in Europa so starke
Schwankungen, dass die Gefahr eines riesigen Blackouts besteht.
\par
Viele Probleme müssen für die Energiewende gelöst werden, eines der wichtigen Kriterien der Energiewende ist aber
das Energie gespart wird. Diese Arbeit wird sich mit einem Teilgebiet der Energiewende befassen.
Es soll aufgezeigt werden, wo in der IT-Branche, speziell im Client- und Serverbereich, Potenzial besteht,
Strom zu sparen. Auch gerade deswegen, weil die IT-Branche von einem enormen Wachstum geprägt ist. Wirtschaftliche Aspekte sollen
berücksichtigt werden, indem der Erfolg des Stromsparens ohne Verzicht auf die Qualität der IT-Infrastruktur erzielt
werden kann. 
\par
Ein Computer ist ein Gerät, dessen Besonderheit darin besteht, dass es durch logische und arithmetische Befehle programmiert
werden kann. In andere Worte gefasst, heisst das, dass die Software die Hardware ansteuert und die Hardware Befehl um Befehl ausführt.
Genau dieser Aspekt soll in der Arbeit abgehandelt werden. Es soll gezeigt werden, wie man durch bessere Software Strom
sparen kann. Im ersten Schritt müssen Messmethoden erforscht und dargestellt werden. Denn nur so kann eine Aussage
gemacht werden, ob ein neues System auch zielführend ist. Erst im zweiten Schritt können dann
Ansätze evaluiert werden, die zum Ziel der Reduktion des Energieverbrauchs dienen können. 
Dabei sollen unterschiedliche Schichten der Software analysiert werden. Es sollen Antworten auf die folgenden Fragen gefunden werden:
Wie kann das Betriebsystem Strom sparen; ist es möglich durch bessere Treiber die Hardware auf Standby zu setzen, wenn sie nicht gebraucht wird
oder kann bereits beim Kompilieren effizenter gearbeitet werden. Am Schluss soll geprüft werden,
ob die Ansätze in der Realität anwendbar und wirtschaflich interessant sind. Es sollen konkrete Beispiele enstehen, wo und wie ein
IT-Betrieb Stromersparnisse erzielen kann.

