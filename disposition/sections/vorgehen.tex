\chapter{Vorgehen}

Der Hauptteil dieser Arbeit wird die Vorbereitung für die Messung sein. Um möglichst viele Störfaktoren
von der Messung auszuschliessen, wird auf ein möglichst einfacher System gesetzt. Deshalb kommt der  SoC
(System on Chip) zur Anwendung. Zusätzlich soll auch die Unregelmässigkeiten des Betriebssystems ausgeschaltet
werden. Aus diesem Grund, soll soweit möglich, das OS ausgelassen werden und direkt auf Bare Metal gearbeitet werden.
Somit lassen sich die Anweisung ganz genau kontrollieren. Für die Messung werden kleine Programme geschrieben, die der Prozessor
exakt einen Befehl mehrmals ausführt. Während der Ausführung wird gemessen und ein Durchschnitts Wert berechnet.
Der SoC ist auf einem Single Board aufgelötet, wobei
die Messung bei dessen Eingangsleistung erfolgen wird. Für das Resultat des EPI muss die Grundleistung abgezählt
werden.
\par
Für das Projekt soll zwei Typen von Prozessoren verwendet werden. Zum einen sollen Messungen von einem RISC-Prozessor
statt finden und zum anderen mit einem CISC-Prozessor. Wo es möglich ist, sollen Vergleich zwischen beiden
Architekturen statt finden.
\par
Bei den durchzuführende Messungen auf den CPU wird nicht nur die unterschiedlichen Operationen untersucht. Es werden 
unterschiedliche auch Experimente erarbeitet. Es soll zum Beispiel eine Aussage gemacht werden, ob die Addieroperation
mit grosse Zahlen mehr Energie verwendet als die selbe Operation mit kleine Zahlen.


