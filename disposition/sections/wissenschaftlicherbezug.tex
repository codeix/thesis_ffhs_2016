\chapter{Wissenschaftlicher Bezug}

\section{Energiepolitik 2050}
Beginnend mit einer Einleitung der Energiepolitik 2050, werde ich den Bezug zu der Entwicklung in der
Informatik schaffen. Dazu werde ich mich auf die Publizierung
"Energiezukunft Schweiz"\cite{eth_energiezukunft_schweiz} von der ETH sowie auf das\cite{swico_datenblatt}
der Swico stützen.

\section{Geschichte der CPUs}
Zur Einleitung will ich in meiner Arbeit auch die Geschichte der Prozessoren präsentieren. Hier soll auch der Bezug
hergestellt werden von RISC und CISC Prozessoren. Hinsichtlich dem Verbrauch spielt der Unterschied eine grosse
Rolle.\cite{stanford_risc_cisc}


\section{Messung}
Für das Experiment müssen Methoden verwendet werden um den Energiebedarf pro Befehlssatz, eines CPU zu messen
\cite{measuring_power_temperature, analysis_circuits, intel_epi}. Für eine optimale Messung müssen die bestehenden Methoden
verfeinert werden oder eigene Entwickelt werden.
\par
Für die Messungen werden spezielle Programme geschrieben, die einen bestimmten Befehlssatz mehrmals, während dem messen
ausführt. Die Auszuführende Befehlssätze für das Experiment werden, anhand dem Datasheet des Herstellers zusammengestellt.
Voraussichtlich wird das Experiment auf einem ARM ARM1176JZF-S\cite{arm_datasheet} und einem Intel® Quark™ SoC
X1000 \cite{intel_datasheet} Prozessor ausgeführt.


