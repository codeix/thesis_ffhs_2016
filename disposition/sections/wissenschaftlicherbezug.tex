\chapter{Wissenschaftlicher Bezug}

\section{Energiepolitik 2050}
Beginnend mit einem Überblick über die Energiepolitik 2050, werde ich den Bezug zur Entwicklung in der
Informatik schaffen. Dazu werde ich mich auf die Publizierung
"Energiezukunft Schweiz"\cite{eth_energiezukunft_schweiz} von der ETH sowie auf die\cite{swico_datenblatt}
der Swico stützen.

\section{Geschichte der CPUs}
Im Anschluss möchte ich in meiner Arbeit auch die Geschichte der Prozessoren präsentieren. Hier wird der Bezug
zwischen RISC- und CISC-Prozessoren hergestellt. Hinsichtlich des Verbrauchs spielt die Unterscheidung zwischen den verschiedenen 
Prozessorarchitekturen eine grosse Rolle.\cite{stanford_risc_cisc}


\section{Messung}
Für das Experiment müssen Methoden verwendet werden, die den Energiebedarf eines CPUs pro Befehlssatz messen
\cite{measuring_power_temperature, analysis_circuits, intel_epi}. Für eine optimale Messung müssen die bestehenden Methoden
verfeinert oder eigene entwickelt werden.
\par
Für die Messungen werden spezielle Programme geschrieben. Diese führen einen bestimmten Befehlssatz während dem Messen mehrmals aus.
Die auszuführenden Befehlssätze für das Experiment werden anhand des Datasheets des Herstellers zusammengestellt.
Voraussichtlich wird das Experiment auf einem ARM ARM1176JZF-S\cite{arm_datasheet} und einem Intel® Quark™ SoC
X1000 \cite{intel_datasheet} Prozessor ausgeführt.


