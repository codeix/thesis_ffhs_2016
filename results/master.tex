\documentclass{../template/ffhsthesis}
\usepackage[a4paper, total={19cm, 26cm}]{geometry}
\usepackage[utf8]{inputenc} % this is needed for umlauts
\usepackage[ngerman]{babel} % this is needed for umlauts
\usepackage[T1]{fontenc}    % this is needed for correct output of umlauts in pdf
\usepackage{pgf,pgffor}
\usepackage{pgfplots}
\usepackage{float}%



\makeatother

\setlength\parindent{0pt}




\begin{document}
% addional configuration from sam
\shorthandoff{"}


\dokumentTyp{Anhang der Bachelor-Thesis}
\studiengang{INF}
\title{Energie in der Informatik}
\subtitle{ durch bessere Software sparen ohne Verzicht} % optional
\titelbild[height=4.55cm,width=15cm]{images/Italy_Alps_and_Mediterranean.jpg}  % optional
\author{Samuel Riolo}
% \date{}
\wohnort{Kerzers}
%\referent{Name des Referenten\\ Titel\\Unterrichtetes Fach}
\referent{Jürg Hofer}
\referent{Jürg Hofer\\ Dipl. El. Ing. ETH\\WING}
%\eingereichtBei{Prof.\ Dr.\ Martin Sutter\\Departement Informatik\\Departementsleiter} 


\maketitle



\startThesis
\chapter{Benchmark results with RaspberryPI}

\tikzset{every picture/.style=thick}


\foreach \fileno in {0,...,100}{
\IfFileExists{raspberrydata/data\fileno.csv}{%
\InputIfFileExists{raspberrydata/data\fileno .tex}{}{}%



\section{\benchmark}
\description
\par
\textit{data\fileno .csv}

\begin{figure}[H]

\foreach \n  in {0,...,2}{%
\begin{minipage}{.33\textwidth}
%
\begin{tikzpicture}
    \begin{axis}[
            axis x line=middle,
            axis y line=middle,
            enlarge y limits=true,
            xmin=0, xmax=130,
            ymin=400, ymax=480,
            width=6cm, height=4cm,     % size of the image
            grid = major,
            grid style={dashed, gray!30},
            ylabel=ampere,
            xlabel=seconds,
            legend style={at={(0.1,-0.1)}, anchor=north}
         ]
          \addplot table [x=time\n, y=amp\n, col sep=comma, mark=none] {raspberrydata/data\fileno .csv};

    \end{axis}
\end{tikzpicture}%



% Example of an array in latex
%\def\names{{"5454.5","Frank","Laura","Joe"}}%
%\pgfmathparse{\names[0]}\pgfmathresult


\begin{tabular}{lcc}
Average:& \pgfmathparse{\average[\n]}\pgfmathresult mA\\
Median:& \pgfmathparse{\median[\n]}\pgfmathresult mA\\
Exec time: & \pgfmathparse{\exectime[\n]}\pgfmathresult ms\\
Exec order:& \pgfmathparse{\run[\n]}\pgfmathresult \\
\end{tabular}

\end{minipage}
}



\end{figure}


}
}





\chapter{Benchmark results with Intel Galileo Gen2}









\end{document}