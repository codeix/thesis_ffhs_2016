\chapter{Idee und Zielsetzung}



Das Interesse am Forschungsgegenstand liegt darin begründet, eine Grundlage zu erstellen, mit deren Hilfe man sparsamere Software schreiben kann. Die Grundidee dieser Engineeringarbeit ist es, aufzuzeigen, wieviel Strom ein Prozessor pro Befehlsatz verbraucht. Durch Messungen werden Daten erstellt, die den EPI (Energy per Instruction)-Verbrauch bestimmen.
\par 
Eine CPU besitzt unveränderliche Schaltkreise, die vom Hersteller meist geheim gehalten werden. Daher ist es nahezu unmöglich, den Verbrauch über die CPU zu verkleinern. Deshalb wurde beschlossen, dass Ziel der Energieeinsparung über optimierte Software zu erreichen. Dazu wird das Optimierungspotenzial der Software, die aus einer Reihenfolge von Befehlssätze besteht, untersucht. Die Schnittstelle zwischen Software und Hardware wird genauer beobachtet. Befehlssätze, die verschwenderisch arbeiten, werden ausfindig gemacht und durch alternative ersetzt. Diese Grundlage könnte in Zukunft dazu beitragen, bessere Kompiler zu bauen, die sparsameren Maschinencode produzieren können. Es könnte so eine Verallgemeinerung von Befehlssätze entstehen von denen man weiss, dass sie besonders Energieeffizient sind. Ist es möglich eine Tabellen zu erstellen mit Befehlssatz und dessen Energieverbrauch, so wäre naheliegend, dass man eine Vorhersage machen kann wie Energieeffizient ein Programm ist.  
\par
Zur Messung des EPI muss eine Messmethodik entwickelt werden, die es Erlaubt eine sehr kurze Zeitspanne, in der der Befehlssatz auf der CPU ausgeführt wird messen zu können. Damit diese unwahrscheinlich kleine Zeitspanne messbar wird, ist die Idee sie mehrere Male nacheinander auszuführen und den Durschnittswert einen einzelnen Befehlssatzes zu errechnen. Die Befehlssätze sollen in einem Miniatur-Programm gekapselt werden. Diese Miniatur-Programme können als Benchmark gesehen werden, die für eine einzelne Messung geschrieben werden. Die Benchmarks sollen in Assembler geschrieben werden damit die grösstmögliche Kontrolle über den Ablauf erreicht wird. Assembler unterscheidet sich zu den Hochsprachen dadurch, dass er sich an Maschinencode orientiert. Der für den Benchmark erstellten Code wird so direkt in Maschinencode übersetzt und durch diese geradewegs Übersetzung können die Befehlsschritte auf dem CPU genau definiert werden. Das erstellen von Benchmarks, die Ausführung und die Messung soll möglichst automatisiert ablaufen, damit eine grössere Menge getestet und anschliessend analysiert werden kann. Die gewählte CPU-Architektur kann einen grossen Einfluss auf den Energieverbrauch haben. Deshalb sollen mindestens zwei Unterschiedliche Architekturen getestet werden. Dabei geht nicht darum herauszufinden welche sparsamer ist, sonder die Unterschiede festzustellen. Ein CPU kann kaum ohne Platine, auf der er aufgelötet ist, betrieben werden. Ein Vollständiger PC wäre für diesen Versuch ungeeignet, weil die umliegende Komponenten auf der Platine unerwünschte Leistungsschwankungen verursachen könnten und die Messung verfälschen würden. Für das Experiment soll eine möglichst einfache Platine verwendet werden, damit die Messung direkt am Stromeingang dieser erfolgen kann. Haben die Komponenten einen regelmässigen Leistungsverbrauch, können die unterschiedlichen Benchmarks, die auf dem CPU ausgeführte werden gemessen werden.
