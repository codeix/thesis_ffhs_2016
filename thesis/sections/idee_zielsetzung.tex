\chapter{Idee und Zielsetzung}



Das Interesse am Forschungsgegenstand liegt darin begründet, eine Grundlage zu erstellen, mit deren Hilfe man sparsamere Software schreiben kann. Die Grundidee dieser Engineeringarbeit ist es, aufzuzeigen, wieviel Strom ein Prozessor pro Befehlsatz verbraucht. Durch Messungen werden Daten erstellt, die den EPI (Energy per Instruction)-Verbrauch bestimmen.
\par 
Eine CPU besitzt unveränderliche Schaltkreise, die vom Hersteller meist geheim gehalten werden. Daher ist es nahezu unmöglich, den Verbrauch über die CPU zu verkleinern. Deshalb wurde beschlossen, dass Ziel der Energieeinsparung über optimierte Software zu erreichen. Dazu wird das Optimierungspotenzial der Software, die aus einer Reihenfolge von Befehlssätze besteht, untersucht. Die Schnittstelle zwischen Software und Hardware wird genauer beobachtet. Befehlssätze, die verschwenderisch arbeiten, werden ausfindig gemacht und durch alternative ersetzt. Diese Grundlage könnte in Zukunft dazu beitragen, bessere Kompiler zu bauen, die sparsameren Maschinencode produzieren können. Darüber hinaus besteht die Möglichkeit, allgemeine Aussagen über den Energieverbrauch von gewissen Befehlssätzen zu generieren. So könnten beispielsweise Tabellen erstellt werden, in denen einzelne Befehlssätze und deren Energieverbrauch aufgeführt sind. Damit wäre nicht nur eine Vorhersage über die Energieeffizienz eines bestimmten Befehlssatzes, sondern eines kompletten Programms realisierbar.  
\par
Zur Messung des EPI-Verbrauchs muss eine Messmethodik entwickelt werden, die es erlaubt die sehr kurze Zeitspanne, in der ein Befehlssatz auf der CPU ausgeführt wird, messen zu können. Damit diese unwahrscheinlich kleine Zeitspanne messbar wird, wurde entschieden, den gleichen Befehlssatz mehrere Male nacheinander auszuführen und anschliessend dessen Durschnittswert zu berechnen. Die Befehlssätze werden dabei in einem Miniatur-Programm gekapselt, welches als Benchmark bezeichnet wird. Bei der Erstellung der Messmethode wird für jede einzelne Messung ein eigener Benchmark in Assembler geschrieben. Die Verwendung von Assembler bezweckt dabei, die grösstmögliche Kontrolle über den Ablauf des Programms. Diese ist deshalb gegeben, weil sich Assembler im Gegensatz zu den Hochsprachen am Maschinencode orientiert. Der für den Benchmark erstellte Code wird durch das Assemblieren geradewegs in Maschinencode übersetzt, wodurch die Befehlsschritte auf der CPU genau definiert werden können. Weitergehend wurde festgelegt, dass das Erstellen der Benchmarks, deren Ausführung und die Messung selbst möglichst automatisiert ablaufen sollen, damit eine grössere Menge an Befehlssätzen getestet und anschliessend analysiert werden kann. Da die ausgewählte CPU-Architektur einen grossen Einfluss auf den Energieverbrauch haben kann, wurde zudem beschlossen, die Messungen mit mindestens zwei unterschiedlichen Architekturen durchzuführen. Dabei geht nicht darum, herauszufinden, welche sparsamer ist, sondern die Unterschiede festzustellen. 

\par

Bei der Erstellung der Messmethodik muss zudem bedacht werden, dass eine CPU kaum ohne Platine, auf der sie aufgelötet ist, betrieben werden kann. Jedoch ist ein vollständiger PC für diesen Versuch ungeeignet, weil die umliegenden Komponenten auf der Platine unerwünschte Leistungsschwankungen verursachen und die Messung verfälschen könnten. Deshalb wird für das Experiment eine möglichst einfache Platine verwendet. Diese weisen weniger Komponenten auf, weshalb der Leistungsverbrauch regelmässiger ist. Dementsprechend kann der Verbrauch der unterschiedlichen Benchmarks, die auf der CPU ausgeführt werden, präziser gemessen werden. Zudem kann die Messung bei einer einfachen Platine direkt am Stromeingang erfolgen, was ebenfalls der Verfälschung der Messresultate vorbeugt.