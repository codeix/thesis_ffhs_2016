\chapter{Idee und Zielsetzung}



In dieser Engineeringarbeit soll durch die zentrale Frage gezeigt werden, wieviel Strom ein Prozessor pro Befehlsatz verbraucht. Durch Messungen werden Daten erstellt, die die EPI (Energy per Instruction) bestimmen. Das Interesse am Forschungsgegenstand liegt darin begründet, eine Grundlage zu erstellen, mit deren Hilfe man sparsamere Software schreiben kann. Dafür soll die Schnittstelle zwischen Software und Hardware genauer beobachtet werden. Es ist anzunehmen, dass man kein Einfluss auf den CPU hat, da dieser unveränderliche Schaltkreise besitzt und vom Herstellt meist geheim gehalten wird. Deshalb ist das Ziel die Energieeinsparung über optimierte Software erfolgen zu erreichen. Die Grundlage könnte helfen bessere Kompiler zu bauen, die im Hinblick auf die Energieeinsparungen, besseren Maschienencode kompilieren können. Es könnte so eine Verallgemeinerung von Befehlssätze entstehen von denen man weiss, dass sie besonders Energieeffizient sind. Ist es möglich eine Tabellen zu erstellen mit Befehlssatz und dessen Energieverbrauch, so wäre naheliegend, dass man eine Vorhersage machen kann wie Energieeffizient ein Programm ist.  
\par
Zur Messung des EPI muss eine Messmethodik entwickelt werden, die es Erlaubt eine sehr kurze Zeitspanne, in der der Befehlssatz auf der CPU ausgeführt wird messen zu können. Damit diese unwahrscheinlich kleine Zeitspanne messbar wird, ist die Idee sie mehrere Male nacheinander auszuführen und den Durschnittswert einen einzelnen Befehlssatzes zu errechnen. Die Befehlssätze sollen in einem Miniatur-Programm gekapselt werden. Diese Miniatur-Programme können als Benchmark gesehen werden, die für eine einzelne Messung geschrieben werden. Die Benchmarks sollen in Assembler geschrieben werden damit die grösstmögliche Kontrolle über den Ablauf erreicht wird. Das erstellen von Benchmarks, die Ausführung und die Messung soll möglichst automatisiert ablaufen, damit eine grössere Menge getestet und anschliessend analysiert werden kann. Die gewählte CPU-Architektur kann einen grossen Einfluss auf den Energieverbrauch haben. Deshalb sollen mindestens zwei Unterschiedliche Architekturen getestet werden. Dabei geht nicht darum herauszufinden welche sparsamer ist, sonder die Unterschiede festzustellen.
