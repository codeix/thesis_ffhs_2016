\section{Resultate}

Die Thesis soll eine Messmethode liefern, die beschreibt, wie man die EPI eines CPUs messen kann. Dabei sollen, wenn möglich,
unterschiedliche CPU-Architekturen berücksichtigt werden. Die Messmethode soll sehr genau beschrieben werden, damit man sie
nachstellen und für eigene Zwecke weiterverwenden kann.
\par
Mit Hilfe der Messmethode sollen interessante Experimente durchgeführt werden. Die Daten aus den Experimenten werden ausgewertet
und in einem Fazit dargestellt.
\par
Die Arbeit soll als Grundlage dienen, weiter in diesem Gebiet zu forschen, um schlussendlich besser verstehen zu können,
welche CPU-Anweisungen energieeffizienter sind.
Anhand dieser Grundlagen sollen sich Massnahmen ableiten lassen, wie auf der Ebene der Hardware Energie eingespart werden kann.
Aufgrund dieses Wissens sollen neue, kreative Ideen entstehen, beispielsweise die Entwicklung und der Bau eines Green-Compilers
oder die Planung eines Energy-Profilers, der bereits beim Schreiben der Software diejenigen Code-Zeilen markiert, welche energetisch
ungeeignet sind.


%Das Resultat meiner Arbeit soll sich nicht nur auf einer theoretische Basis bewegen,
%sonder klar aufzeigen wo Handlungsbedarf besteht. Aus der Arbeit soll sich klare
%Handlungsempfehlungen entnehmen. Die Resultate sollen sich realistisch auf ein IT-Betrieb
%umsetzen lassen. Dabei soll ein IT-Betrieb Energie sparen können, ohne Verzicht auf die Qualität.

