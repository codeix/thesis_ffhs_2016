\chapter{Einleitung}



In der heutigen Zeit ist die nachhaltige Produktion von Energie und ein schonender Umgang mit dieser Ressource ein wichtiges Anliegen der Öffentlichkeit. Speziell die IT-Branche ist von einem grossen Wachstum geprägt. Immer mehr Tätigkeiten werden automatisiert und bestehende Systeme ausgebaut. Dies verursacht einen immer grösseren Stromverbrauch im IT-Bereich. Ein Ziel der Energiepolitik 2050 ist die Senkung des immer weiter steigenden Energieverbrauchs.
\par
Um in der IT-Branche Energie zu sparen, können verschiedene Ansätze gewählt werden. Beispielsweise besteht die Möglichkeit bei Inaktivität automatisch in den Ruhemodus zu wechseln oder die Prozessoren durch sogenannte Sparmodi, insbesondere dynamische Taktfrequenzen oder das kurzzeitige Ein- und Ausschalten der Rechnereinheit, zu optimieren. Diese Sparansätze zielen jeweils nur auf den Hard- oder Softwarebereich ab. Sie berücksichtigen jedoch nicht die besondere Funktionsweise eines jeden Computers, wo die Software die Hardware, also ein Programm einen Prozessor, ansteuert. Anders im Rahmen dieser Arbeit, in welcher das Sparpotenzial bereichsübergreifend erforscht wird.





















\begin{comment}
\section{Übersicht}

Die Energiepolitik ist auf der ganzen Welt ein grosses Thema und wird heftig diskutiert. Immer mehr Leute
sind der Überzeugung, dass eine Energiewende unumgäglich ist. Dabei wird die Energiewende von politischen, aber auch
von wirtschaftlichen Interessen beeinflusst. Die Energiewende beinhaltet aber nicht nur
das Produzieren von erneuerbaren Energien, sondern viele unterschiedliche Teilaspekte. Für die Energiewende
müssen Energieressourcen gespeichert werden, damit sie bei Bedarf genutzt werden können. Wird Strom über
Solaaranlagen in den Haushalten produziert, so muss die Energie ins Netz zurückfliessen. Dafür müssen
Hochspannungsleitungen so verändert werden, dass sie Strom in beide Richtungen fliessen lassen können.
Die heute verwendeten Energienanlagen, wie Windmühlen oder Solaaranlagen, produzieren in Europa so starke
Schwankungen, dass die Gefahr eines riesigen Blackouts besteht.
\par
Viele Probleme müssen für die Energiewende gelöst werden. Eines der wichtigsten Kriterien der Energiewende ist aber,
dass Energie gespart wird. Diese Arbeit wird sich mit einem Teilgebiet der Energiewende befassen.
Es soll aufgezeigt werden, wo in der IT-Branche, speziell im Client- und Serverbereich, Potenzial besteht,
Strom zu sparen. 

% todo Überarbeiten, ausführlicher
Auch gerade deswegen, weil die IT-Branche von einem enormen Wachstum geprägt ist. Wirtschaftliche Aspekte sollen
berücksichtigt werden, indem der Erfolg des Stromsparens ohne Verzicht auf die Qualität der IT-Infrastruktur erzielt
werden kann. 
\par
% todo Zusammenfassung der Arbeit, nicht Beschreibung, was ich machen werde
Ein Computer ist ein Gerät, dessen Besonderheit darin besteht, dass es durch logische und arithmetische Befehle programmiert
werden kann. In andere Worte gefasst, heisst das, dass die Software die Hardware ansteuert und die Hardware Befehl um Befehl ausführt.
Genau dieser Aspekt soll in der Arbeit abgehandelt werden. Es soll eine Basis gebildet werden, die aufzeigt, wie man durch bessere Software Strom
sparen kann. Die Arbeit wird sich darauf konzentrieren, den Energiebedarf pro Befehlssatz eines Prozessors zu analysieren.
Es soll eine Messmethoden erforscht und entwickelt werden, um den Strombedarf eines CPU zu untersuchen. Die Untersuchung soll
auf unterschiedlichen SoC (System on Chip) Architekturen erfolgen, damit Vergleiche erstellt werden können.
\par
Ziel der Arbeit ist es, Daten zu erhalten, die Informationen über den Energieverbrauch einzelner Befehlssätze liefern. Diese Daten
könnte man zum Beispiel auf Workstations oder Server hochrechnen. Für CPU-Befehle, die sich eher verschwenderisch auf den
Strombedarf auswirken, könnte man Alternativen finden. Man könnte sich zum Beispiel eine Green-Flag für den GNU C Compiler
vorstellen, der "grünen Maschinencode" erstellt.

\end{comment}


% Denn nur so kann eine Aussage gemacht werden, ob ein neues System auch zielführend ist. Erst im zweiten Schritt können dann
% Ansätze evaluiert werden, die zum Ziel der Reduktion des Energieverbrauchs dienen können. 
% Dabei sollen unterschiedliche Schichten der Software analysiert werden. Es sollen Antworten auf die folgenden Fragen gefunden werden:
% Wie kann das Betriebsystem Strom sparen; ist es möglich durch bessere Treiber die Hardware auf Standby zu setzen, wenn sie nicht gebraucht wird
% oder kann bereits beim Kompilieren effizenter gearbeitet werden. Am Schluss soll geprüft werden,
% ob die Ansätze in der Realität anwendbar und wirtschaflich interessant sind. Es sollen konkrete Beispiele enstehen, wo und wie ein
% IT-Betrieb Stromersparnisse erzielen kann.


