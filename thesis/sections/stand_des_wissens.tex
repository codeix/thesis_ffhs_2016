\chapter{Stand des Wissens und der Technik}
\section{Funktionsweise eines Prozessor}

Ein Prozessor ist eine universelle Rechnermaschine, die sich durch eine definierte Reihe von Anweisungen programmieren lässt. Zu den arithmetische Anweisungen gehört der Zugriff auf Speicheradressen und Sprünge innerhalb der Abfolge der Anweisungen.
Bereits der britische Mathematiker Alan Turing konnte aufzeigen, dass ein universelles Berechnungsmodell möglich ist wenn ein Rechner neben Speicher zugriff auch Sprünge besitzt\cite{Hoffmann2014l}.




\section{Unterschiede CISC und RISC CPUs}



\section{Energy Storage in a Capacitor}
