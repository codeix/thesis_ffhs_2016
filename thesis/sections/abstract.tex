\begin{zusammenfassung}

Die Energiepolitik ist auf der ganzen Welt ein grosses Thema und wird heftig diskutiert. Immer mehr Leute sind der Überzeugung, dass eine Energiewende unumgänglich ist. Eines der wichtigsten Kriterien der Energiewende ist, dass Energie gespart wird. Dies kann sowohl durch die Senkung des Energieverbrauchs an sich als auch durch Optimierung der Produktion und Speicherung von Energie erreicht werden.
\par
Diese Arbeit befasst sich mit der Option, den Energieverbrauch zu senken. Es wird aufgezeigt, wo in der IT-Branche, speziell im Client- und Serverbereich, Potenzial besteht, Strom zu sparen.
\par
Im Rahmen der Arbeit wird eine Messmethode erforscht und entwickelt, um den Strombedarf eines CPU-Prozessors zu untersuchen. Damit kann dessen Energiebedarf pro Befehlssatz gemessen und analysiert werden. Die Untersuchung erfolgt auf zwei unterschiedlichen SoC-Board, nämlich eines mit RISC- und eines mit CISC-Architektur.

\textit{Todo: Einfügen der Resultat und Schlüsse, evtl. noch mehr zur Untersuchungsmethode (z.B. wie wird der Stromverbrauch gemessen?)}


\end{zusammenfassung}

\begin{abstract} 

-- Just an dirty Google translation as placeholder, I will fix it later --
Energy policy is a big issue around the world and is hotly debated. More and more people are the
Believes that energy policy is unumgäglich. One of the main criteria of the energy transition, however, the
Energy is saved. This work will focus on a subsection of the energy transition. It should be pointed out,
where is the IT industry, especially in the client and server area, potential to save power. economic aspects
should be considered by the success in efficiency, without sacrificing the quality of the IT infrastructure
can be achieved.
\par
The work focused on it, the energy consumption per instruction set of a processor
analyze. It will be researched and developed a measurement methods to examine the current requirement of a CPU.
The investigation is carried out on different SoC (System on Chip) architectures to allow comparisons created
can be.

\end{abstract}
