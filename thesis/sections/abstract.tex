\begin{zusammenfassung}

Die Energiepolitik ist auf der ganzen Welt ein grosses Thema und wird heftig diskutiert. Diese Arbeit befasst sich mit der Zielsetzung, den Energieverbrauch zu senken. Es wird eine Messmethode erforscht und entwickelt, um den Strombedarf eines CPU-Prozessors zu untersuchen. Mit dieser wird dessen Energiebedarf pro Befehlssatz gemessen und analysiert. Dabei hat sich herausgestellt, dass generelle Aussagen, wie ein Befehlssatz, der Energie verschwendet, optimiert werden soll, nicht leicht zu treffen sind. Punktuelle Optimierungen an einer bestimmten Stelle des Sourcecodes können jedoch ermittelt werden. Zusätzlich konnte festgestellt werden, dass die gemessene Differenzen zwischen den einzelnen Befehlssätze von unterschiedlichen Faktoren abhängig sind. Dazu gehören etwa die CPU-Architektur oder die Wortlänge des Operanden, der berechnet wird. Dabei hat sich gezeigt, dass die grösste Energieeinsparung durch Befehlssätze mit einer geringeren Anzahl CPU-Zyklen erreicht werden kann. 





\end{zusammenfassung}

\begin{abstract} 

-- Just an dirty Google translation as placeholder --
\par
Energy policy is a big issue around the world and is hotly debated. More and more people are the
Believes that energy policy is unumgäglich. One of the main criteria of the energy transition, however, the
Energy is saved. This work will focus on a subsection of the energy transition. It should be pointed out,
where is the IT industry, especially in the client and server area, potential to save power. economic aspects
should be considered by the success in efficiency, without sacrificing the quality of the IT infrastructure
can be achieved.
\par
The work focused on it, the energy consumption per instruction set of a processor
analyze. It will be researched and developed a measurement methods to examine the current.....

\end{abstract}
