\begin{zusammenfassung}

Die Energiepolitik ist auf der ganzen Welt ein grosses Thema und wird heftig diskutiert. Diese Arbeit befasst sich mit der Zielsetzung, den Energieverbrauch zu senken. Es wird eine Messmethode erforscht und entwickelt, um den Strombedarf eines CPU-Prozessors zu untersuchen. Mit dieser wird dessen Energiebedarf pro Befehlssatz gemessen und analysiert. Dabei hat sich herausgestellt, dass generelle Aussagen, wie ein Befehlssatz, der Energie verschwendet, optimiert werden soll, nicht leicht zu treffen sind. Punktuelle Optimierungen an einer bestimmten Stelle des Sourcecodes können jedoch ermittelt werden. Zusätzlich konnte festgestellt werden, dass die gemessene Differenzen zwischen den einzelnen Befehlssätze von unterschiedlichen Faktoren abhängig sind. Dazu gehören etwa die CPU-Architektur oder die Wortlänge des Operanden, der berechnet wird. Dabei hat sich gezeigt, dass die grösste Energieeinsparung durch Befehlssätze mit einer geringeren Anzahl CPU-Zyklen erreicht werden kann. 





\end{zusammenfassung}

\begin{abstract}

Energy policy is an important and controversially debated topic all around the 
globe, with an increasing number of people and enterprises believing that it 
is inevitable to increase energy efficiency. 
Thus the goal of this thesis is to lower power consumption. 
A method to measure power consumption of a CPU is developed and analysed, 
which can be used to 
measure the energy usage per instruction. 
The results show that it is hard to specify generic rules on how to optimize a 
certain energy wasting instruction set. 
However, punctual optimizations in the source code which is translated to 
these instructions can be identified. 
In addition to that, several factors having an impact on power consumption 
where found. Among them where the CPU architecture and the word length or 
precision of the operands to be calculated. 
Choosing instruction sets that require fewer CPU cycles turned out to be the 
most energy efficient optimization. 
 

\end{abstract}
