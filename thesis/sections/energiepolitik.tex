\chapter{Energiepolitik 2050}

\section{Politische Entscheidungen in der Schweiz}
Durch den Beschluss am 2011 des Bundesrat und Parlament wurde ein Grundsatzentscheid getroffen,
schrittweise aus der Kernenergie auszusteigen\cite{bfe_energiestrategie}. 
\par
Die Stillegung der fünf Kernkraftwerke in der Schweiz, sollen am Ende ihrer sicherheitstechnischen
Betriebsdauer erfolgen. Diese Entscheidung und insbesondere auch die internationale Veränderungen 
im der Energieversorgung bewirkt eine starke Wende in das schweizerische Energieumfeld.
Am 4. September 2013 wurde vom Bundesrat und Parlament das erste Massnahmenpaket verabschiedet, um
die Sicherheit der Energieversorgung in der Schweiz zu garantieren. Das Massnahmenpaket setzt in erster
Line auf eine ausgewogene Ausschöpfung der vorhandene Potenziale der Wasserkraft und der neuen erneuerbaren
Energien. Am 8. Dezember 2014 wurde das Massnahmenpaket vom Nationalrat und am 23. September 2015 vom Ständerat
angenommen. Differenzen werden zur Zeit zwischen den Räten bereinigt bevor diese nochmals über die
gesamte Vorlage abstimmt. 
\par
Das Parlament hat bereits mit einer Gesetzesänderung (pa.lv. 12.400) verabschiedet, dass am Anfang 2014 in
Kraft getreten ist. Dadurch werden der Ausbau der erneuerbaren Energien gefördert. Auch beinhaltet
diese Gesetzesänderung ein Aktionsplan für eine stärkere Energieforschung.

\section{Stromnachfrage 2010 - 2050}


\section{}


