\chapter{Energiepolitik 2050}



\section{Energiepolitische Entscheidungen in der Schweiz\cite{bfe_energiestrategie}}
Durch den Beschluss des Bundesrates und des Parlaments im Mai 2011 wurde der Grundsatzentscheid gefällt,
schrittweise aus der Kernenergie auszusteigen. 
\par
Die Stilllegung der fünf Kernkraftwerke in der Schweiz soll am Ende ihrer sicherheitstechnischen
Betriebsdauer erfolgen. Diese Entscheidung und insbesondere auch die internationalen Veränderungen 
im Energiesektor bewirken eine grundlegende Wende in der schweizerischen Energiepolitik.
Am 4. September 2013 wurde vom Bundesrat und Parlament das erste Massnahmenpaket verabschiedet, um
die Sicherheit der Energieversorgung in der Schweiz auch zukünftig garantieren zu können. Das Massnahmenpaket "setzt in erster
Linie auf eine konsequente Erschliessung der vorhandenen Energieeffizienzpotentiale und in zweiter Linie auf eine ausgewogene Ausschöpfung der vorhandenen Potenziale der Wasserkraft und der neuen erneuerbaren Energien." Das Massnahmenpaket wurde in der Wintersession 2014 vom Nationalrat und in der Herbstsession 2015 vom Ständerat
angenommen. Derzeit findet das Differenzbereinigungsverfahren in den beiden Räten statt. Anschliessend werden diese schliesslich über die gesamte Vorlage in der Schlussabstimmung befinden. 
\par
Das Parlament hat bereits mit eine Gesetzesänderung (pa.lv. 12.400) verabschiedet, nach welcher der Ausbau der erneuerbaren Energien vermehrt gefördert wird. Diese trat Anfang 2014 in Kraft. Zusätzlich wurde ebenso ein Aktionsplan für eine verstärkte Energieforschung gutgeheissen.

\section{Stromnachfrage 2010 - 2050\cite{eth_energiezukunft_schweiz}}

Aufgrund der Gesetzgebungsarbeiten zur Energiepolitik 2050 des Bundes und dem damit verbundenen Ziel, die Energieversorgung der Schweiz nachhaltig gewährleisten zu können, hat die ETH eine Studie zur Entwicklung des Energiebedarfs der Schweiz und der Möglichkeiten diesen zu decken, durchgeführt. Sie erstellten eine Prognose der Stromnachfrage über einen Zeithorizont von 40 Jahren. Dazu verwendeten sie die Parameter Bevölkerungswachstum, Pro-Kopf-Einkommen und Stromintensität. Die Erstellung einer genauen Prognose stellte sich aufgrund des zu berücksichtigenden langen Zeitraumes generell als sehr heikel dar. Dennoch liessen sich plausible Werte für das Jahr 2050 abbilden.
\par
Heute rechnet man mit einem Stromverbrauch von zirka 63 TWh pro Jahr, wobei Netzverluste von etwa 7\% einbezogen, hingegen Speicherpumpenverluste nicht berücksichtigt werden. Für das Jahr 2050 geht man davon aus, dass sich Extremwerte von 60 TWh bis über 100TWh pro Jahr abzeichnen werden.
\par
Den schrittweisen Ausstieg aus der Atomkraft zu kompensieren und die gleichzeitig neu auftretenden Ansprüche bezüglich des Strombedarfs abzudecken, scheint aufgrund des geschätzten Verbrauchs im Jahr 2050 schwierig, jedoch nicht unmöglich zu sein. Für eine erfolgreiche Umwandlung des Energiesystems der Schweiz über die nächsten 40 Jahre müssen sowohl Investitionen in die Forschung getätigt, als auch der Einsatz von Photovoltaik, Biomasse, Wind und Geothermie intensiviert werden.
Ebenso wird es erforderlich sein, den Ausbau von Stauseen und Pumpspeicherkraftwerken aktiv zu fördern. Dieser Prozess wird sich zwar nicht völlig problemlos gestalten, ist jedoch im Rahmen der technologischen Gegebenheiten in der Schweiz grundsätzlich umsetzbar und ökonomisch verkraftbar.
\par
Des Weiteren führt die CO\textsubscript{2}-Reduktion und die damit kombinierte Stromgewinnung aus erneuerbaren Energien ebenfalls dazu, dass die Importmengen von Energie um 65\% verkleinert werden können. Dadurch wird die Schweiz hinsichtlich der Stromversorgung autonomer und die Versorgungssicherheit des Landes verbessert sich zusehends.
 


\section{Ziele für die Energieeinsparung 2050}






