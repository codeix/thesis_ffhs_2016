\chapter{Energiepolitik 2050}


Ziel der Thesisarbeit im Endeffekt, ist es durch bessere Software Strom zu sparen. Deshalb wird hier dem Leser
eine Übersicht in sehr stark zusammengefasster Form geboten, um die Zusammenhänge und Bewegungen auserhalb der
IT-Branche besser zu verstehen.

\section{Energiepolitische Entscheidungen in der Schweiz}
Durch den Beschluss am 2011 des Bundesrat und Parlament wurde ein Grundsatzentscheid getroffen,
schrittweise aus der Kernenergie auszusteigen\cite{bfe_energiestrategie}. 
\par
Die Stillegung der fünf Kernkraftwerke in der Schweiz, sollen am Ende ihrer sicherheitstechnischen
Betriebsdauer erfolgen. Diese Entscheidung und insbesondere auch die internationale Veränderungen 
im der Energieversorgung bewirkt eine starke Wende in das schweizerische Energieumfeld.
Am 4. September 2013 wurde vom Bundesrat und Parlament das erste Massnahmenpaket verabschiedet, um
die Sicherheit der Energieversorgung in der Schweiz zu garantieren. Das Massnahmenpaket setzt in erster
Line auf eine ausgewogene Ausschöpfung der vorhandene Potenziale der Wasserkraft und der neuen erneuerbaren
Energien. Am 8. Dezember 2014 wurde das Massnahmenpaket vom Nationalrat und am 23. September 2015 vom Ständerat
angenommen. Differenzen werden zur Zeit zwischen den Räten bereinigt bevor diese nochmals über die
gesamte Vorlage abstimmt. 
\par
Das Parlament hat bereits mit einer Gesetzesänderung (pa.lv. 12.400) verabschiedet, dass am Anfang 2014 in
Kraft getreten ist. Dadurch werden der Ausbau der erneuerbaren Energien gefördert. Auch beinhaltet
diese Gesetzesänderung ein Aktionsplan für eine stärkere Energieforschung.

\section{Stromnachfrage 2010 - 2050}

Die Prognosen der Stromnachfrage über ein Zeithorizont von 40 Jahren ist generell sehr schwer\cite{eth_energiezukunft_schweiz}.
Dennoch bilden sich plausible Wert für das Jahr 2050 ab. Dabei werden die Parameter Bevölkerungswachstum,
Pro-Kopf-Einkommen und Stromintensität aufgrund der Trends, berücksichtigt.
\par
Heute geht man von einem Stromverbrauch von 63 TWh pro Jahr aus (inkl. Netzverluste von etwa 7\% aber
ohne Speicherpumpenverluste). Für das Jahr 2050 geht man davon aus, dass sich die Extremwerte von
60 TWh/a bis über 100TWh/a abzeichnen werden.
\par
Der schrittweise Ausstieg aus der Atomkraft und gleichzeitig die neuen Ansprüche des Strombedarfs abzudecken,
scheint für das 2050 schwierig aber nicht unmöglich zu sein. Die Transformation des Energiesystem in einem
Zeithorizont von mehrere Jahrzehnten ist im Grundsatz technologisch machbar und wirtschaftlich verkraftbar.
Für den Erfolg muss in die Forschung und Einsatz von Photovoltaik, Biomasse, Wind und Geothermie investiert werden.
Auch der Ausbau von Stauseen, als Pumpspeicherkraftwerken wird dafür nötig sein.
\par
Die CO\textsubscript{2}-Reduzierung und die damit kombinierte Stromerzeugung aus erneuerbare Energien, führt auch dazu,
dass die importiert mengen um 65\% reduziert werden. Dabei wird ein wesentlicher Beitrag geleistet, dass die Schweiz
hinsichtlich der Stromversorgung unabhängiger wird und eine besser  Versorgungssicherheit des Landes gewährleistet wird.
 


\section{Ziele für die Energieeinsparung 2050}






