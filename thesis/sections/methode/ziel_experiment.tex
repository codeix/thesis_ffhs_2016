\section{Auswahl und Beschreibung der Messmethode}

Die Idee der hier erstellte Messmethode, ist der CPU über ein Benchmark einen bestimmt Befehlssatz mehrmals durchlaufen zu lassen und die gesamte Leistung des Gerät zu messen. Optimal wäre nur die Leistung des CPU zu messen. Dafür müsste aber der CPU vom Board ausgelötet werden und eine spezielle dafür gebaute Messvorrichtung gebaut werden. Dazu kommt, dass nicht jedes Board oder Chip alle nötigen Datenblätter für die Beschreibung der Pins frei verfügbar sind. Deshalb wird ein Soc eingesetzt und die gesamte Leistung gemessen. SoC sind verfügbar auf kleine Entwickler-Boards, die keine Lüfter, Laufwerke oder andere Verbraucher die  auf Hinsicht der Leistung störende Unregelmässigkeiten aufweisen würden. Die für diese Arbeit verwendete Board werden im nächsten Kapitel beschrieben. Es wird davon Ausgegangen das während Durchführung des Benchmark, die Komponenten auf dem Board abgesehen des CPU, vernachlässigbar kleine Leistungsschwankungen aufweisen. Die Messung erhält somit eine Grundleistung die vom Resultat abgezogen werden muss. Die Speisung erfolgt über ein Spannungsregler der, der Schaltung eine konstante Spannung liefert. Zwischen dem Spannungsregler und der Schaltung ist ein Amperemeter(Multimeter) geschaltet der die Messung vornimmt.
\par
Der eigentliche Kern des Benchmark besteht aus gezielten und kurzen Assembler-Zeilen. Der Assemblercode bewirkt eine Schleife über einen zu testenden Befehlssatz. Damit das Ergebnis über einen längeren Zeitraum gemessen werden kann, wird die Schleife um eine Grössenordnung von 100 Millionen nacheinander wiederholt. Dies erstellt ein Zeitraum für die Messung von ca. einer halbe Minute, auf einem 800MHz CISC-Prozessor.
\par
Während dem durchlaufen des Benchmark darf die Ausführung nicht gestört werden. OS, Zeitscheibe etc...Bli bla blup