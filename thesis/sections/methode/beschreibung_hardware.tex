\section{Beschreibung der Hardware}

\subsection{Intel Galileo Gen2}

\begin{wrapfigure}{r}{0.5\textwidth}
\centering
\includegraphics[scale=0.5]{images/iot_galileo.png}
\caption{Intel Galileo Gen2\cite{intel_galileo_image}}
\label{fig:Intel Galileo Gen2}
\end{wrapfigure}

Das Galileo Developer Board\cite{intel_datasheet_galileo} der zweite Generation von
Intel ist ein SoC und besitzt ein Intel® Quark™ SoC X1000 Prozessor. Die Architektur des 32bit Prozessor basiert auf x86\cite{intel_datasheet} und somit ist das Galileo
Board eins der wenigen, auf das ein CISC-Prozessor verbaut wurde. Die Hardware
besitzt kein Videoausgang. Es ist also nur möglich über ein RS232 oder über Ethernet
eine Verbindung zum System zu erstellen. Die GPIO Ein- und Ausgänge sind über den
PCI-Bus verbunden. Somit ist es sehr schwer ohne passender Treiber oder
Softwareschicht die GPIO anzusprechen. Im Vergleich sind die GPIO des RaspberryPI
direkt über eine Speicheradressen ansprechbar.
\par
Von Intel wird ein angepasstes Betriebssystem als fix fertiges Image angeboten. Dieses System basiert auf
die Linux Distribution Yocto. Es sind aber auch inoffizielle Debian Distribution im
Internet erhältlich. Der Vorteil von Debian gegenüber die offizielle Version, ist die
grössere Verbreitung und die damit verbundene Hilfestellungen im Internet. Das Cross-Compiling eines Kernelmodul scheint so einfacher als auf der Yocto Distribution.


\subsection{RaspberryPi}


\begin{wrapfigure}{r}{0.5\textwidth}
\centering
\includegraphics[scale=0.4]{images/raspberry-pi-2.png}
\caption{Raspberry Pi 1 A\cite{raspberry_image}}
\label{fig:Raspberry Pi 1 A}
\end{wrapfigure}


Als zweites Experimentier-Board verwende ich das Raspberry Pi Model B\cite{raspberry_foundation}. Das Board wird von der Raspberry Pi Foundation betrieben. Das Modell besitzt ein Broadcom BCM2835\cite{broadcom_datasheet} Chip. Der darin verbaute Prozessor wurde von der Firma ARM unter dem Name ARM1176JZFS\cite{arm_datasheet} (ARM11) spezifiziert. Die Software muss für die Architektur ARMv6 kompiliert werden. Im Gegensatz zum Galileo Board von Intel ist der Prozessor ein RISC-CPU. Beide Board haben eine 32bit Adressierung.
\par
Eine Remote-Verbindung lässt sich über eine Serielle-Schnittstelle oder über Ethernet realisieren. Durch den besitz des Videoausgang und den USB-Anschluss, würde sich das Raspberry Pi auch direkt über Monitor und Tastatur bedienen lassen. Was allerdings für das Experiment nicht nötig sein wird. Die GPIO sind direkt über die Speicheraderessirrung ansprechbar. Somit wäre eine Erweiterung des Benchmark der in Assembler geschrieben ist, durchaus denkbar. So könnte auf einfache Weise weitere Messungen über die GPIO erfolgen. Das offizielle Betriebssystem der Foundation ist ein Dabian basiertes OS mit dem Name Raspbian. Zusätzlich werden als Third-Party Produkte weitere unterschiedliche Betriebssysteme zur Verfügung gestellt. 











