\chapter{Fazit}

Im Bezug auf die Umwelt und das streben nach einer Energieeinsparung, zeigt diese Arbeit, dass es möglich ist einzelne Befehlssätze zu messen und auszuwerten. Es zeigt aber auch wie unterschiedlich die Verschiedene Architekturen sind und wie sie sich deshalb die Benchmarks komplett anderes auswirken. Es ist anzunehmen, dass die erzeugten Daten der Benchmarks nur für ein CPU-Typ gültig sind und es nicht möglich ist eine Schlussfolgerung auf andere CPUs, der selben Modellreihe zu schliessen. Somit bleibt eine Verallgemeinerung im Bezug auf den Leistungsverbrauch der Befehlssätze aus. Die Arbeit zeigt auch, dass grosse Differenzen existieren innerhalb eines Befehlssatz. Dabei spielt es eine grosse Rolle welche Zahlen eine Operation behandeln muss, wie grosse die zu behandelten Wortbreite ist oder in welchen Reihenfolge Befehlssätze durchlaufen müssen. Um ein Versuch einer Verallgemeinerung auf einer einzelne Hardware zu erstellen, müssten viel grössere Mengen an Benchmarks durchgeführt werden. Damit möglichst jeder Fall abgedeckt wird und sich daraus eine Verallgemeinerung aus dieser hohe Komplexität der Gestaltung der Befehlssätze erstellen würde. Eine andere Möglichkeit, ist ein bestimmten Algorithmus in einer Applikation zu Optimieren und dafür geeignete Benchmarks erstellen. Dieses Vorgehen hätte den Vorteil, dass man sich nicht auf eine aufwändige Verallgemeinerung stützt, sondern sich auf eine bestimmte stelle im Programmcode der Applikation konzentriert. Die Optimierung würde dort statt finden wo vorhersehbar ist, dass ein Programm an einer bestimmter Stelle sehr viele Male durchläuft.
\par
% Best Case scenario

\par
% Best Performance decrease cpu cycles
Die grösste Energieeinsparung erhält man wann man Befehlssätze verwendet, die möglichst geringe CPU-Zyklen für die Ausführung benötigen. Dies haben die Berechnungen aus den Auswertungen auf dem Galileo und Raspberry Board gezeigt, wenn man die Energie aus Leistung und Durchlaufzeit nimmt. Ein Befehlssatz der nur ein CPU-Zyklus anstelle von zwei verwendet hat folglich eine Einsparung von fast 50\%. Dies hat als neben Effekt nicht nur die Energieeinsparung zur Folge, sonder auch ein schnelleres abarbeiten einer Prozedur. Und deshalb sollte dies Ziel immer angestrebt werden.
\par
Bei allen Optimierungen bleibt die Frage offen, ob es für einen Energie verschwenderischen Befehlssatz sich eine gleichwertige alternative finden lässt. Diese Frage hängt sehr stark nach der Anforderung des Programmcode ab. Es ist durchaus denkbar das zweifache eines Wert durch eine Multiplikation oder durch eine binäre Verschiebung mit dem selben Resultat zu erreichen. Falls die Wortlänge nicht 16 Bit überschreitet könnte anstelle des Befehlssatz \texttt{add} auch \texttt{sadd16} für die Erzeugung des selben Resultates verwendet werden. Es ist also möglich alternative Befehlssätze zu finden, es verlangt aber sehr starke engineering Arbeit und muss für jeden Anwendungsfall individuell gelöst werden.












%Vergleich mit ähnliche Papers (z.B. Energiemessung durch thermische Kamera)
%Das Resultat in Sätze gefasst








